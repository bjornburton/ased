\input cwebmac

\M{1}
With an emergency generator connected through an interlocked load-center, it's
hard to tell when the Ancillary Service has been restored. Switching back to
Main shuts everything down.
The obvious method is to measure the voltage at the main-breaker's input.
The safety concern is that it's not breaker-protected making for a massive
fault-current,
should insulation be breached or the circuit shorted.
Also, installation is somewhat involved, having to tap into live lines and
providing some form of isolation.

The obvious solution is to have a high-impedance connection.
A small capacitance will do.
Simple capacitive coupling can be had with a ``gimmick''; a technique used
since the
1920s.
This may be several turns of THHN around the large-gage insulated incomming
line.
Since the voltage is with respect to neutral, and neutral is bonded to ground,
just the one wire is needed.
No need to mess with conductors, just coupling to the electric field through
the insulator.

With this signal, a circuit and be built to detect the difference between
having AC and having no AC,
and provide a signal to indicate that state.
The signal provided to the generator-operator could be a lamp or buzzer.

The line-voltage is $\pm$ 170~V peak, with respect to ground.
The peaks will be about 8.3~ms apart.
The signal will be much less, depending on the circuit's input impedance and
capacitance.

The circuit would need a high input impedance, so as to see a strong enough
signal.
The circuit would need a reference to ground to compare to.
The input may need a bit of protection from line transients, which could pass
trough the gimmick.

\fi

\M{2}
I had seven Ada Fruit Trinkets just laying around. They use the Atmel
ATTINY85 processor. The analog inputs are about 100~M$\Omega$. Not great, but
I think it should be good enough. If we can muster 1~pf of gimmick, we will
have $\ {1 \over 2 \pi f_c} $ of $X_c$. Ohms law indicates
$100e6 {170 \over  {(2pi*60*1e-12)^{-1} + 100e6)} } = 6.16 $ volts peak,
ignoring
input pin capacitance. The steering diodes will keep the analog innards safe
since the current is so low. Supply voltage at "BAT" is 5.5 to 16~V and it has
a red LED on-board.

\fi

\M{3}Here's the code

\Y\B\C{************************************* ASED Version 1.1.0r Ancillary
Service Electric Detector 2015-01-01 Bjorn Burton  Just for fun.
*************************************}\6
\8\#\&{include} \.{"ased.h"}\par
\fi

\M{4}At the beginning the I/O is configured.
\Y\B\&{int} \\{main}(\&{void})\1\1 $\{{}$\C{ set the led port direction }\6
$\.{DDRB}\MRL{{\OR}{\K}}(\T{1}\LL\.{LED\_RED\_DD}){}$;\C{ set the siren port
direction }\6
${}\.{DDRB}\MRL{{\OR}{\K}}(\T{1}\LL\.{SIREN\_DD}){}$;\C{ enable pin change
interrupt for clear-button}\6
${}\.{PCMSK}\MRL{{\OR}{\K}}(\T{1}\LL\.{PCINT3}){}$;\C{ General interrupt Mask
register for clear-button}\6
${}\.{GIMSK}\MRL{{\OR}{\K}}(\T{1}\LL\.{PCIE}){}$;\par
\fi

\M{5}The LED is set ''assuming'' that there will be an AC signal.

\Y\B\C{ turn the led on }\6
\\{ledcntl}(\.{ON});\par
\fi

\M{6}Here the timer and comparator are setup.
\Y\B\C{ set up the nowave timer }\6
\\{initnowavetimer}(\,);\par
\fi

\M{7}
The Trinket runs at a speedy 8 MHz so the slow 60 Hz signal is no issue.
One could use the ADC but that doesn't make too much sense as the input may
spend a lot of time cliped.
The inbuilt comparator seems like the right choice, for now.
/* set up the wave-event comparator */
initwavedetector();

\fi

\M{8}Of course, any interrupt function requires that bit I "Global Interrupt
Enable"
is set; usualy done through calling sei().
\Y\B\C{ Global Int Enable }\6
\\{sei}(\,);\par
\fi

\M{9}
Rather than burning loops, waiting for something to happen for 16 ms, the sleep
mode can be used.
Interrupts are used to wake it.
\Y\B\C{ configure sleep\_mode() to go to ``idle''. Idle allows     the counters
and comparator to continue during sleep. }\6
\.{MCUCR}${}\MRL{\AND{\K}}\CM(\T{1}\LL\.{SM1});{}$\6
${}\.{MCUCR}\MRL{\AND{\K}}\CM(\T{1}\LL\.{SM0}){}$;\par
\fi

\M{10}
This is the loop that does the work.
\Y\B\&{for} ( ;  ; \,)\SHC{ forever }\6
${}\{{}$\1\6
\&{static} \&{unsigned} \&{char} \\{nowaves}${}\K\.{WAVETHRESHOLD};{}$\6
\&{static} \&{unsigned} \&{int} \\{armwait}${}\K\.{ARMTHRESHOLD}{}$;\C{
hold-off to minimise noise susceptibilty }\7
\\{waveholdoff}(\,);\C{ now we wait in idle for any interrupt event }\6
\\{sleep\_mode}(\,);\C{ some interrupt was detected! Let's see which one }\6
\&{if} ${}(\\{f\_state}\AND(\T{1}\LL\.{WAVES})){}$\5
${}\{{}$\1\6
${}\\{nowaves}\K(\\{nowaves})\?\\{nowaves}-\T{1}:\T{0};{}$\6
\&{if} ${}(\R\\{nowaves}{}$)\SHC{ ancilliary electric service restored }\6
${}\{{}$\1\6
\\{ledcntl}(\.{ON});\6
\&{if} ${}(\\{f\_state}\AND(\T{1}\LL\.{ARM})){}$\1\5
\\{chirp}(\.{ON});\2\6
${}\.{TCNT1}\K\.{TIMESTART}{}$;\SHC{ reset the timer }\6
\4${}\}{}$\2\6
${}\\{f\_state}\MRL{\AND{\K}}\CM(\T{1}\LL\.{WAVES}){}$;\SHC{reset int flag }\6
\4${}\}{}$\2\6
\&{else} \&{if} ${}(\\{f\_state}\AND(\T{1}\LL\.{NOWAVES})){}$\5
${}\{{}$\1\6
\\{ledcntl}(\.{OFF});\6
${}\\{nowaves}\K\.{WAVETHRESHOLD};{}$\6
\\{chirp}(\.{OFF});\SHC{ ASE dropped, stop alarm chirp }\6
${}\\{armwait}\K(\\{armwait})\?\\{armwait}-\T{1}:\T{0};{}$\6
\&{if} ${}(\R\\{armwait}\W\CM\\{f\_state}\AND(\T{1}\LL\.{ARM})){}$\1\5
${}\\{f\_state}\MRL{{\OR}{\K}}(\T{1}\LL\.{ARM}){}$;\C{ at this time the only
way to disarm is a power cycle }\2\6
${}\\{f\_state}\MRL{\AND{\K}}\CM(\T{1}\LL\.{NOWAVES}){}$;\SHC{reset int flag }\6
\4${}\}{}$\2\6
\4${}\}{}$\2\6
\&{return} \T{0};\SHC{ it's the right thing to do! }\6
$\}{}$\par
\fi

\M{11}
Siren function will arm after a 10 minute power-loss; that is,
the Trinket is running for a full 10 minutes without seeing AC at pin \#2.
Once armed, siren will chirp for 100 ms at a 5 second interval,
only while AC is present.
It may be disarmed, stopping the chirp, by pressing a button or power-cycle.
\Y\B\C{ Alarm Pulsing function }\6
\&{void} \\{chirp}(\&{char} \\{state})\1\1\2\2\6
${}\{{}$\1\6
\&{static} \&{unsigned} \&{char} \\{count}${}\K\.{CHIRPLENGTH};{}$\7
${}\\{count}\K(\\{count})\?\\{count}-\T{1}:\.{CHIRPPERIOD};{}$\6
\&{if} ${}(\\{count}>\.{CHIRPLENGTH}\V\\{state}\E\.{OFF}){}$\1\5
\\{sirencntl}(\.{OFF});\2\6
\&{else}\1\5
\\{sirencntl}(\.{ON});\2\6
\4${}\}{}$\C{ simple led control }\2\7
\&{void} \\{ledcntl}(\&{char} \\{state})\1\1\2\2\6
${}\{{}$\1\6
${}\.{PORTB}\K\\{state}\?\.{PORTB}\OR(\T{1}\LL\.{LED\_RED}):\.{PORTB}\AND\CM(%
\T{1}\LL\.{LED\_RED});{}$\6
\4${}\}{}$\C{ simple siren control }\2\7
\&{void} \\{sirencntl}(\&{char} \\{state})\1\1\2\2\6
${}\{{}$\1\6
${}\.{PORTB}\K\\{state}\?\.{PORTB}\OR(\T{1}\LL\.{SIREN}):\.{PORTB}\AND\CM(\T{1}%
\LL\.{SIREN});{}$\6
\4${}\}{}$\C{ configure the no-wave timer }\2\7
\&{void} \\{initnowavetimer}(\&{void})\1\1\2\2\6
${}\{{}$\C{ set a very long prescal of 16384 counts }\1\6
${}\.{TCCR1}\K((\T{1}\LL\.{CS10})\OR(\T{1}\LL\.{CS11})\OR(\T{1}\LL\.{CS12})\OR(%
\T{1}\LL\.{CS13})){}$;\C{ Timer/counter 1 f\_overflow interupt enable }\6
${}\.{TIMSK}\MRL{{\OR}{\K}}(\T{1}\LL\.{TOIE1});{}$\6
\4${}\}{}$\C{ configure the the wave detection comparator }\2\6
\&{void} \\{initwavedetector}(\&{void})\1\1 $\{{}$\par
\fi

\M{12}
The ideal input AN1, PB1, is connected to the LED in the Trinket!
That's not a big issue since the ADC's MUX may be used.
That MUX may address PB2, PB3, PB4 or PB5. Of those, PB2, PB3 and PB4 are
available.
Since PB3 and PB4 are use for USB, PB2 makes sense here.
This is marked ``\#2'' on the Trinket.
PB2 connects the the MUX's ADC1.
Use of the MUX is selected by setting bit ACME of port ADCSRB. ADC1 is set by
setting bit MUX0 of register ADMUX
\Y\B\C{ Setting bit ACME of port ADCSRB to enable the MUX input ADC1 }\6
\.{ADCSRB}${}\MRL{{\OR}{\K}}(\T{1}\LL\.{ACME}){}$;\C{ ADC1 is set by setting
bit MUX0 of register ADMUX }\6
${}\.{ADMUX}\MRL{{\OR}{\K}}(\T{1}\LL\.{MUX0}){}$;\par
\fi

\M{13}Disable digital input buffers at AIN[1:0] to save power. This is done by
setting AIN1D and AIN0D in register DIDR0.

\Y\B\C{ Disable digital inputs to save power }\6
\.{DIDR0}${}\MRL{{\OR}{\K}}((\T{1}\LL\.{AIN1D})\OR(\T{1}\LL\.{AIN0D})){}$;\par
\fi

\M{14}
Both comparator inputs have pins but AIN0 can be connected to a reference of
1.1 VDC, leaving the negative input to the signal. The ref is selected by
setting bit ACBG of register ACSR.
\Y\B\C{ Connect the + input to the band-gap reference }\6
\.{ACSR}${}\MRL{{\OR}{\K}}(\T{1}\LL\.{ACBG}){}$;\par
\fi

\M{15}
It can be configured to trigger on rising, falling or toggle (default) by
clearing/setting bits ACIS[1:0] also on register ACSR.
There is no need for toggle, and falling is selected by simply setting ACIS1.
\Y\B\C{ Trigger on falling edge only }\6
\.{ACSR}${}\MRL{{\OR}{\K}}(\T{1}\LL\.{ACIS1}){}$;\par
\fi

\M{16}
To enable this interrupt, set the ACIE bit of register ACSR.
\Y\B\C{ Enable the analog comparator interrupt }\6
\.{ACSR}${}\MRL{{\OR}{\K}}(\T{1}\LL\.{ACIE});$ $\}{}$\par
\fi

\M{17}
A timer is needed to to encompase some number of waves so it can clearly
discern on from off.
The timer is also interrupt based. The timer is set to interrupt at overflow.
It could overflow within about 1/2 second.
Over the course of that time, 25 to 30 comparator interrupts are expected.
When the timer interrupt does occour, the LED is switched off.
Comparator Interrupts are counted and at 15 the timer is reset and the LED is
switched on.
\Y\B\C{ Wave detection Hold-Off or debounce }\6
\&{void} \\{waveholdoff}(\,)\1\1\2\2\6
${}\{{}$\C{ Disable the analog comparator interrupt }\1\6
${}\.{ACSR}\MRL{\AND{\K}}\CM(\T{1}\LL\.{ACIE});{}$\6
\\{\_delay\_us}(\.{WAVEHOLDOFFTIME});\C{ Enable the analog comparator interrupt
}\6
${}\.{ACSR}\MRL{{\OR}{\K}}(\T{1}\LL\.{ACIE});{}$\6
\4${}\}{}$\C{ Timer ISR }\2\7
\.{ISR}(\\{TIMER1\_OVF\_vect})\1\1\2\2\6
${}\{{}$\1\6
${}\\{f\_state}\MRL{{\OR}{\K}}(\T{1}\LL\.{NOWAVES});{}$\6
\4${}\}{}$\2\par
\fi

\M{18}
The event can be checked by inspecting (then clearing) the ACI bit of the ACSR
register but the vector ANA\_COMP\_vect is the simpler way.

\Y\B\C{ Comparator ISR }\6
\.{ISR}(\\{ANA\_COMP\_vect})\1\1\2\2\6
${}\{{}$\1\6
${}\\{f\_state}\MRL{{\OR}{\K}}(\T{1}\LL\.{WAVES});{}$\6
\4${}\}{}$\C{ Clear Button ISR }\2\7
\.{ISR}(\\{PCINT0\_vect})\1\1\2\2\6
${}\{{}$\1\6
\&{if} ${}(\.{PORTB}\AND(\T{1}\LL\.{ARMCLEAR})){}$\1\5
${}\\{f\_state}\MRL{\AND{\K}}\CM(\T{1}\LL\.{ARM});{}$\2\6
\4${}\}{}$\2\par
\fi


\inx
\fin
\end
