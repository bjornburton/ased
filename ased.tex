\input cwebmac

% ased

\nocon % omit table of contents
\datethis % print date on listing


\M{1}
ASED Version 1.1.0r
Ancillary Service Electric Detector
Bjorn Burton

With an emergency generator connected through an interlocked load-center, it's
hard to tell when the \PB{\\{Ancillary}\\{Service}} has been restored.
Switching back to Main requiers shutting  everything down for a moment.
It would be good to know if main is live before swicking back. The obvious
method is to measure the voltage at the main-breaker's input.
The safety concern is that it's not breaker-protected making for a massive
fault-current,
should insulation be breached or the circuit shorted.
Also, installation of a simple meter is somewhat involved, having to tap into
live lines and, idealy, providing some form of isolation.

The obvious solution is to have a high-impedance connection very near to the
source.
A small capacitance would do.
Simple capacitive coupling can be had with a ``gimmick''; a technique used
since the 1920s.
This may be several turns of THHN around the large-gage insulated incomming
line.
Since the voltage is with respect to neutral, and neutral is bonded to ground,
just the one wire is needed.
No need to mess with live conductors, just coupling to the electric field
through the insulation already present. Installation still has some risk, but
much less.

With this signal, a circuit and be built to detect the difference between
having AC and having no AC,
and provide a signal to indicate that state.
The signal provided to the generator-operator could be a lamp or buzzer.

The line-voltage is $\pm$ 170~V peak, with respect to ground.
The peaks will be about 8.3~ms apart.
The signal will be much less, depending on the circuit's input impedance and
capacitance.

The circuit would need a high input impedance, so as to see a strong enough
signal.
The circuit would need a reference to ground to compare to.
The input may need a bit of protection from line transients, which could pass
trough the gimmick.

I had seven Ada Fruit Trinkets just laying around. They use the Atmel
ATTINY85 processor. The analog inputs are about 100~M$\Omega$. Not great, but
I think it should be good enough. If we can muster 1~pf of gimmick, we will
have $\ {1 \over 2 \pi f_c} $ of $X_c$. Ohms law indicates
$100e6 {170 \over  {(2pi*60*1e-12)^{-1} + 100e6)} } = 6.16 $ volts peak,
ignoring
input pin capacitance. The steering diodes will keep the analog innards safe
since the current is so low. Supply voltage at "BAT" is 5.5 to 16~V and it has
a red LED on-board.

In use, the AC signal goes to the pin  marked ``\#2'' on the Trinket.
The LED port is marked ``\#1''.
The Siren port is marked ``\#0''.
Clear, should it be implemented, is on ``\#3''.

\Y\B\4\D$\.{F\_CPU}$ \5
\T{8000000\$U\$L}\par
\B\4\D$\.{ARMCLEAR}$ \5
\.{PORTB3}\SHC{ Trinket's Clear is  pin \#3 }\par
\B\4\D$\.{ARMCLEAR\_DD}$ \5
\.{DDB3}\C{ Boolean }\par
\B\4\D$\.{ON}$ \5
\T{1}\par
\B\4\D$\.{OFF}$ \5
\T{0}\par
\B\4\D$\.{SET}$ \5
\T{1}\par
\B\4\D$\.{CLEAR}$ \5
\T{0}\C{ Flags for f\_state }\par
\B\4\D$\.{NOWAVES}$ \5
\T{2}\SHC{ no ASE detected for some time }\par
\B\4\D$\.{WAVES}$ \5
\T{1}\SHC{ ASE detected }\par
\B\4\D$\.{ARM}$ \5
\T{0}\SHC{ ARM for Alarm }\C{ Parameters }\par
\B\4\D$\.{WAVETHRESHOLD}$ \5
\T{15}\C{ Number of wave before considering the service on. Range to 255 }\C{
About 250 ms. Don't take too long or time will run out }\par
\B\4\D$\.{TIMESTART}$ \5
\T{12}\SHC{ preset for the timer counter. Range to 255 */ }\C{ Prescaler is set
to clk/16484. 0.5 seconds *(8e6/16384) is 244.14. 256-244 = 12, leaving 500 ms
to time-out }\par
\B\4\D$\.{WAVEHOLDOFFTIME}$ \5
\T{100}\SHC{ Range to 255 }\C{ hold-off time in us for wave detection }\par
\B\4\D$\.{ARMTHRESHOLD}$ \5
\T{1200}\SHC{ Range to 65535 }\C{ alarm arm delay in nowave counts}\C{ chirp
parameters for alarm }\par
\B\4\D$\.{CHIRPLENGTH}$ \5
\T{7}\SHC{ number of waves long }\par
\B\4\D$\.{CHIRPPERIOD}$ \5
\T{200}\SHC{ number of waves long }\par
\B\4\D$\.{LED\_RED}$ \5
\.{PORTB1}\SHC{ Trinket's LED and pin \#1 }\par
\B\4\D$\.{LED\_RED\_DD}$ \5
\.{DDB1}\C{ siren port }\par
\B\4\D$\.{SIREN}$ \5
\.{PORTB0}\SHC{ Trinket's Siren and pin \#0 }\par
\B\4\D$\.{SIREN\_DD}$ \5
\.{DDB0}\par
\Y\B\8\#\&{include} \.{<avr/io.h>}\SHC{ need some port access }\6
\8\#\&{include} \.{<util/delay.h>}\SHC{ need to delay }\6
\8\#\&{include} \.{<avr/interrupt.h>}\SHC{ have need of an interrupt }\6
\8\#\&{include} \.{<avr/sleep.h>}\SHC{ have need of sleep }\6
\8\#\&{include} \.{<stdlib.h>}\par
\fi

\M{2}\B\X2:Prototypes\X${}\E{}$\6
\&{void} \\{ledcntl}(\&{char} \\{state});\SHC{ LED ON and LED OFF }\6
\&{void} \\{sirencntl}(\&{char} \\{state});\SHC{ alarm siren control }\6
\&{void} \\{chirp}(\&{char} \\{state});\SHC{ alarm siren modulation }\par
\U4.\fi

\M{3}
The f\_state global variable needs the type qualifier `volatile' or
optimization may eliminate it.
f\_state is just a simple bit-flag that keeps track what has been handled.

\Y\B\4\X3:Global variables\X${}\E{}$\6
\&{volatile} \&{unsigned} \&{char} \\{f\_state}${}\K\T{0}{}$;\par
\U4.\fi

\M{4}
Atmel pins default as simple inputs so they need to be configured to use them
for output.
Additionaly, we need the clear button to wake the device through an interrupt.
\Y\B\X2:Prototypes\X\6
\X3:Global variables\X\6
\&{int} \\{main}(\&{void})\1\1 $\{$ \X13:Initialize pin outputs and inputs\X\par
\fi

\M{5}
The LED is set, meaning `on', assuming that there is an AC signal.
The thought is that it's better to say that there is AC, when there isn't, as
opposed to the converse.
\Y\B\C{ turn the led on }\6
\\{ledcntl}(\.{ON});\par
\fi

\M{6}
Here is the timer and comparator are setup.
\Y\B\C{ set up the nowave timer }\6
\X15:Initialize the no-wave timer\X\par
\fi

\M{7}
The Trinket runs at relativly speedy 8 MHz so the slow 60 Hz signal is no
issue.
One could use the ADC but that doesn't make too much sense as the input may
spend a lot of time cliped.
We just need to know when the signal changes.
The inbuilt comparator seems like the right choice, for now.

\Y\B\C{ set up the wave-event comparator }\6
\X16:Initialize the wave detection\X\par
\fi

\M{8}
Of course, any interrupt function requires that bit ``Global Interrupt Enable''
is set; usualy done through calling sei().
\Y\B\C{ Global Int Enable }\6
\\{sei}(\,);\par
\fi

\M{9}
Rather than burning loops, waiting for something to happen for 16 ms, the sleep
mode is used.
The specific type of `sleep' is `idle'.
Interrupts are used to wake it.
\Y\B\X17:Configure to wake upon interrupt\X\par
\fi

\M{10}
This is the loop that does the work. It should spend most of its time in \PB{%
\\{sleep\_mode}}, comming out at each interrupt event.
The ISRs alter the bits in \PB{\\{f\_state}}.

\Y\B\&{for} ( ;  ; \,)\SHC{ forever }\6
$\{$ \&{static} \&{unsigned} \&{char} \\{nowaves}${}\K\.{WAVETHRESHOLD};{}$\6
\&{static} \&{unsigned} \&{int} \\{armwait}${}\K\.{ARMTHRESHOLD}{}$;\par
\fi

\M{11}
Now we wait in idle for any interrupt event
\Y\B\\{sleep\_mode}(\,);\par
\fi

\M{12}
Some interrupt has been  detected! Let's see which one
\Y\B\&{if} ${}(\\{f\_state}\AND(\T{1}\LL\.{WAVES})){}$\5
${}\{{}$\1\6
${}\\{nowaves}\K(\\{nowaves})\?\\{nowaves}-\T{1}:\T{0}{}$;\SHC{ countdown to 0,
but not lower }\6
\&{if} ${}(\R\\{nowaves}{}$)\SHC{ ancillary electric service restored }\6
${}\{{}$\1\6
\\{ledcntl}(\.{ON});\6
\&{if} ${}(\\{f\_state}\AND(\T{1}\LL\.{ARM}){}$)\SHC{ now we annunciate this
fact }\1\6
\\{chirp}(\.{ON});\2\6
${}\.{TCNT1}\K\.{TIMESTART}{}$;\SHC{ reset the timer }\6
\4${}\}{}$\2\6
${}\\{f\_state}\MRL{\AND{\K}}\CM(\T{1}\LL\.{WAVES}){}$;\SHC{reset int flag
since actions are complete }\6
\4${}\}{}$\2\6
\&{else} \&{if} ${}(\\{f\_state}\AND(\T{1}\LL\.{NOWAVES})){}$\5
${}\{{}$\1\6
\\{ledcntl}(\.{OFF});\6
${}\\{nowaves}\K\.{WAVETHRESHOLD};{}$\6
\\{chirp}(\.{OFF});\SHC{ ASE dropped, stop alarm chirp }\6
${}\\{armwait}\K(\\{armwait})\?\\{armwait}-\T{1}:\T{0}{}$;\SHC{ countdown to 0,
but not lower }\6
\&{if} ${}(\R\\{armwait}\W\CM\\{f\_state}\AND(\T{1}\LL\.{ARM})){}$\1\5
${}\\{f\_state}\MRL{{\OR}{\K}}(\T{1}\LL\.{ARM}){}$;\C{ at this time the only
way to disarm is a power cycle }\2\6
${}\\{f\_state}\MRL{\AND{\K}}\CM(\T{1}\LL\.{NOWAVES}){}$;\SHC{reset int flag }\6
\4${}\}{}$\2\6
\X18:Hold-off all interrupts\X $\}$ \&{return} \T{0};\SHC{ it's the right thing
to do! }\6
$\}{}$\par
\fi

\M{13}\B\X13:Initialize pin outputs and inputs\X${}\E{}$\6
${}\{{}$\C{ set the led port direction }\1\6
${}\.{DDRB}\MRL{{\OR}{\K}}(\T{1}\LL\.{LED\_RED\_DD}){}$;\C{ set the siren port
direction }\6
${}\.{DDRB}\MRL{{\OR}{\K}}(\T{1}\LL\.{SIREN\_DD}){}$;\C{ enable pin change
interrupt for clear-button}\6
${}\.{PCMSK}\MRL{{\OR}{\K}}(\T{1}\LL\.{PCINT3}){}$;\C{ General interrupt Mask
register for clear-button}\6
${}\.{GIMSK}\MRL{{\OR}{\K}}(\T{1}\LL\.{PCIE});{}$\6
\4${}\}{}$\2\par
\U4.\fi

\M{14}
Siren function will arm after a 10 minute power-loss; that is,
the Trinket is running for a full 10 minutes without seeing AC at pin \#2.
Once armed, siren will chirp for 100 ms at a 5 second interval,
only while AC is present. In fact it is called with each AC cycle interrupt so
that \PB{\.{CHIRPLENGTH}} and \PB{\.{CHIRPPERIOD}} are defined a multiples of
${1 \over Hz}$.
It may be disarmed, stopping the chirp, by pressing a button or power-cycle.

\Y\B\&{void} \\{chirp}(\&{char} \\{state})\1\1\2\2\6
${}\{{}$\1\6
\&{static} \&{unsigned} \&{char} \\{count}${}\K\.{CHIRPLENGTH};{}$\7
${}\\{count}\K(\\{count})\?\\{count}-\T{1}:\.{CHIRPPERIOD};{}$\6
\&{if} ${}(\\{count}>\.{CHIRPLENGTH}\V\\{state}\E\.{OFF}){}$\1\5
\\{sirencntl}(\.{OFF});\2\6
\&{else}\1\5
\\{sirencntl}(\.{ON});\2\6
\4${}\}{}$\2\7
\&{void} \\{ledcntl}(\&{char} \\{state})\1\1\2\2\6
${}\{{}$\1\6
${}\.{PORTB}\K\\{state}\?\.{PORTB}\OR(\T{1}\LL\.{LED\_RED}):\.{PORTB}\AND\CM(%
\T{1}\LL\.{LED\_RED});{}$\6
\4${}\}{}$\C{ simple siren control }\2\7
\&{void} \\{sirencntl}(\&{char} \\{state})\1\1\2\2\6
${}\{{}$\1\6
${}\.{PORTB}\K\\{state}\?\.{PORTB}\OR(\T{1}\LL\.{SIREN}):\.{PORTB}\AND\CM(\T{1}%
\LL\.{SIREN});{}$\6
\4${}\}{}$\2\par
\fi

\M{15}
A timer is needed to to encompase some number of waves so it can clearly
discern on from off.
The timer is also interrupt based. The timer is set to interrupt at overflow.
It could overflow within about 1/2 second.
Over the course of that time, 25 to 30 comparator interrupts are expected.
When the timer interrupt does occour, the LED is switched off.
Comparator Interrupts are counted and at 15 the timer is reset and the LED is
switched on.

\Y\B\4\X15:Initialize the no-wave timer\X${}\E{}$\6
${}\{{}$\SHC{set a very long prescale of 16384 counts }\1\6
${}\.{TCCR1}\K((\T{1}\LL\.{CS10})\OR(\T{1}\LL\.{CS11})\OR(\T{1}\LL\.{CS12})\OR(%
\T{1}\LL\.{CS13})){}$;\C{ Timer/counter 1 f\_overflow interrupt enable }\6
${}\.{TIMSK}\MRL{{\OR}{\K}}(\T{1}\LL\.{TOIE1});{}$\6
\4${}\}{}$\2\par
\U6.\fi

\M{16}
The ideal input AN1 (PB1), is connected to the LED in the Trinket!
That's not a big issue since the ADC's MUX may be used.
That MUX may address PB2, PB3, PB4 or PB5. Of those, PB2, PB3 and PB4 are
available.
Since PB3 and PB4 are use for USB, PB2 makes sense here.
This is marked ``\#2'' on the Trinket.
PB2 connects the the MUX's ADC1.
Use of the MUX is selected by setting bit ACME of port ADCSRB. ADC1 is set by
setting bit MUX0 of register ADMUX


Disable digital input buffers at AIN[1:0] to save power. This is done by
setting AIN1D and AIN0D in register DIDR0.

Both comparator inputs have pins but AIN0 can be connected to a reference of
1.1 VDC, leaving the negative input to the signal. The ref is selected by
setting bit ACBG of register ACSR.


It can be configured to trigger on rising, falling or toggle (default) by
clearing/setting bits ACIS[1:0] also on register ACSR.
There is no need for toggle, and falling is selected by simply setting ACIS1.


To enable this interrupt, set the ACIE bit of register ACSR.

\Y\B\4\X16:Initialize the wave detection\X${}\E{}$\6
${}\{{}$\C{ Setting bit ACME of port ADCSRB to enable the MUX input ADC1 }\1\6
${}\.{ADCSRB}\MRL{{\OR}{\K}}(\T{1}\LL\.{ACME}){}$;\C{ ADC1 is set by setting
bit MUX0 of register ADMUX }\6
${}\.{ADMUX}\MRL{{\OR}{\K}}(\T{1}\LL\.{MUX0}){}$;\C{ Disable digital inputs to
save power }\6
${}\.{DIDR0}\MRL{{\OR}{\K}}((\T{1}\LL\.{AIN1D})\OR(\T{1}\LL\.{AIN0D})){}$;\C{
Connect the + input to the band-gap reference }\6
${}\.{ACSR}\MRL{{\OR}{\K}}(\T{1}\LL\.{ACBG}){}$;\C{ Trigger on falling edge
only }\6
${}\.{ACSR}\MRL{{\OR}{\K}}(\T{1}\LL\.{ACIS1}){}$;\C{ Enable the analog
comparator interrupt }\6
${}\.{ACSR}\MRL{{\OR}{\K}}(\T{1}\LL\.{ACIE});{}$\6
\4${}\}{}$\2\par
\U7.\fi

\M{17}
Setting these bits configure sleep\_mode() to go to ``idle''.
Idle allows the counters and comparator to continue during sleep.

\Y\B\4\X17:Configure to wake upon interrupt\X${}\E{}$\6
${}\{{}$\1\6
${}\.{MCUCR}\MRL{\AND{\K}}\CM(\T{1}\LL\.{SM1});{}$\6
${}\.{MCUCR}\MRL{\AND{\K}}\CM(\T{1}\LL\.{SM0});{}$\6
\4${}\}{}$\2\par
\U9.\fi

\M{18}\B\X18:Hold-off all interrupts\X${}\E{}$\6
${}\{{}$\C{ Disable the analog comparator interrupt }\1\6
${}\.{ACSR}\MRL{\AND{\K}}\CM(\T{1}\LL\.{ACIE});{}$\6
\\{\_delay\_us}(\.{WAVEHOLDOFFTIME});\C{ Enable the analog comparator interrupt
}\6
${}\.{ACSR}\MRL{{\OR}{\K}}(\T{1}\LL\.{ACIE});{}$\6
\4${}\}{}$\2\par
\U12.\fi

\M{19}
This is the ISR for the main timer.
When this overflows it generaly means the ASE has been off for a while.
\Y\B\C{ Timer ISR }\6
\.{ISR}(\\{TIMER1\_OVF\_vect})\1\1\2\2\6
${}\{{}$\1\6
${}\\{f\_state}\MRL{{\OR}{\K}}(\T{1}\LL\.{NOWAVES});{}$\6
\4${}\}{}$\2\par
\fi

\M{20}
The event can be checked by inspecting (then clearing) the ACI bit of the ACSR
register but the vector ANA\_COMP\_vect is the simpler way.

\Y\B\C{ Comparator ISR }\6
\.{ISR}(\\{ANA\_COMP\_vect})\1\1\2\2\6
${}\{{}$\1\6
${}\\{f\_state}\MRL{{\OR}{\K}}(\T{1}\LL\.{WAVES});{}$\6
\4${}\}{}$\2\par
\fi

\M{21}
This ISR responds to the Clear button.
\Y\B\C{ Clear Button ISR }\6
\.{ISR}(\\{PCINT0\_vect})\1\1\2\2\6
${}\{{}$\1\6
\&{if} ${}(\.{PORTB}\AND(\T{1}\LL\.{ARMCLEAR})){}$\1\5
${}\\{f\_state}\MRL{\AND{\K}}\CM(\T{1}\LL\.{ARM});{}$\2\6
\4${}\}{}$\2\par
\fi

\M{22}
Done

\fi


\inx
\fin
\end
